\section{乘法逆元}
\subsection{简介}
\begin{frame}[fragile]{乘法逆元}{简介}
  \textbf{乘法逆元}:如果一个线性同余方程 $ax \equiv 1 \pmod b$,则 $x$ 称为 $a \bmod b$ 的逆元,记作 $a^{-1}$。\\
  
  \pause
  \vspace{0.5cm}
  \textbf{用处}:
  \begin{itemize}
    \item 求解$\frac{A}{B}\bmod{P}$
    \item 避免浮点运算,提高精度
    \item 某些题目的要求
  \end{itemize}
  
  \pause 
  \vspace{0.5cm}
  \textbf{求解}:
  \begin{itemize}
    \item 快速幂方法
    \item 拓展欧几里得方法
    \pause
    \item 依据个人喜好
  \end{itemize}

\end{frame}

\subsection{单个求解}
\begin{frame}[fragile]{乘法逆元}{求解单个}
  \textbf{快速幂}
  \begin{itemize}
    \item \textbf{事实}:
    $$
    \begin{aligned}
       &\;& a^{b-1} &\equiv& 1 \pmod p \\
      &\Rightarrow& a\cdot a^{b-2} &\equiv& 1 \pmod p \\ 
      &\Rightarrow& x &\equiv& a^{b-2} \pmod p
    \end{aligned}
    $$
    \pause 
    \item \textbf{算法实现}:
    \begin{lstlisting}
inline int qpow(long long a, int b) {
  int ans = 1;
  a = (a % p + p) % p;
  for (; b; b >>= 1) {
    if (b & 1) ans = (a * ans) % p;
    a = (a * a) % p;
  }
  return ans;
}
// int inv = qpow(a, p - 2);
    \end{lstlisting}
    \pause 
    \item 时间复杂度:$O(\log{n})$
  \end{itemize}
\end{frame}

\begin{frame}[fragile]{乘法逆元}{求解单个}
  \textbf{拓展欧几里得}:
  \begin{itemize}
    \item \textbf{事实}:
    $$
    \begin{aligned}
      ax\equiv 1 \pmod{p}\\
      ax=(-y)\cdot p+1\\
      \text{求解:}ax+py=1
    \end{aligned}
    $$

    \pause
    \item \textbf{算法实现}:
    \begin{lstlisting}
int exgcd(int a, int b, int& x, int& y) {
  if (b == 0) {
    x = 1, y = 0;
    return a;
  }
  int g = exgcd(b, a % b, y, x);
  y -= a / b * x;
  return g;
}
// exgcd(a, p, x, y);
// int inv = (x % p + p) % p;
  \end{lstlisting}

  \pause 
  \item 时间复杂度:$O(\log{n})$
\end{itemize}
\end{frame}

\subsection{多个求解}
\begin{frame}[fragile]{乘法逆元}{求解$1\sim n$}
  线性时间复杂度求出 $1,2,...,n$ 中每个数关于 $p$ 的逆元。
  
  \pause 
  \begin{proof}
    \textbf{事实}:$\forall p \in \Z,\space 1 \times 1 \equiv 1 \pmod p$恒成立,即$\pmod{p}$意义下$1^{-1}=1$。\\
    
    \pause 
    \vspace{0.1cm}
    \textbf{思路}:通过$1\sim i-1$的逆元在$O(1)$的复杂度下求解$i^{-1}$。\\
    
    \pause 
    \vspace{0.1cm}
    设:
    $$
    k = \lfloor \frac{p}{i} \rfloor, \space j = p \bmod i, \space\text{则}p = ki + j
    $$
    
    \pause 
    我们有:
    $$
    \begin{aligned}
      &p=ki+j &\equiv& 0 &\pmod p&\\
      \pause 
      &ki\cdot{i^{-1}j^{-1}}+j\cdot{i^{-1}j^{-1}} &\equiv& 0\cdot{i^{-1}j^{-1}} &\pmod{p}& \\
      &k\cdot{j^{-1}}+i^{-1} &\equiv& 0 &\pmod{p}& \\
      \pause
      &i^{-1} &\equiv& -k\cdot{j^{-1}} &\pmod{p}& \\
      &i^{-1} &\equiv& -k\cdot \lfloor \frac{p}{i} \rfloor &\pmod{p}&
    \end{aligned}
    $$
  \end{proof}
\end{frame}