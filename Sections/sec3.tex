\section{素数}
\subsection{素数}
\begin{frame}{素数基础}
  \textbf{定义}:只能被1和自身整除的数称为素数,又称为质数
  \begin{itemize}
    \item 1既不是素数,也不是合数
    \item 2是最小的素数,也是唯一的偶素数
  \end{itemize}
  \vspace{0.5cm}
  \pause 
  \textbf{基本性质}:
  \begin{itemize}
    \item 素数计数函数$\pi(x)$:$\pi(x)\sim \frac{x}{\ln{x}}$
    \pause
    \item 素数的分布:$1e9$范围内,任意两个相邻的素数差不超过$400$
    \pause
    \item \textbf{威尔逊定理}:$(p-1)!\equiv -1\pmod{p}$
  \end{itemize}
\end{frame}

\begin{frame}[fragile]{素数的判定}
  \begin{itemize}
    \item 暴力枚举所有可能的因子即可
    \item 事实:如果$x|a$,那么$\frac{a}{x}|a$。
    \item 判定算法:
\begin{lstlisting}
inline bool isPrime(ll n) {
  if (n < 2) return false;
  for (int i = 2; i <= sqrt(n) + 1; ++i) {
    if (!(n % i)) return false;
  }
  return true;
}
\end{lstlisting}
  \item 时间复杂度:$O(\sqrt{n})$
  \pause
  \item Miller-Rabin 素性测试:$O(k\log^3{n})$
  \item \sout{我不会}
  \end{itemize}
\end{frame}


\subsection{筛法}
\begin{frame}[fragile]{筛法}
  筛法用于求出$1\sim n$中所有的素数。
  \begin{itemize}
    \item 事实:一个数$x$的任意整数倍一定是素数
    \pause
    \item 对每个数,将其倍速标记为\textbf{非素数},那么没被标记的数,就是素数
    \pause
    \item 算法实现:
\begin{lstlisting}
const int MAXN = 1e6 + 5;
bool inp[MAXN], prime[MAXN], cnt;
inline void getPrime(int n) {
    memset(inp, 0, sizeof(inp));
    for (int i = 2; i <= n; ++i) {
        if (!inp[i]) prime[++cnt] = i;
        for (int j = 2; i * j <= n; ++j)
        inp[i * j] = true;
    }
}
\end{lstlisting}
  \item 时间复杂度:$O(n\log{n})$
  \end{itemize}
\end{frame}

\begin{frame}[fragile]{埃拉托斯特尼筛法}
  \begin{itemize}
    \item 只需要对素数的倍数进行标记就好了
    \pause
    \item 算法实现:
\begin{lstlisting}
inline void getPrime(int n) {
    memset(inp, 0, sizeof(inp));
    for (int i = 2; i <= n; ++i) {
        if (inp[i]) continue; 
        prime[++cnt] = i;
        for (int j = 2; i * j <= n; ++j)
            inp[i * j] = true;
    }
}
\end{lstlisting}
    \item 时间复杂度:$O(n\log{\log{n}})$
  \end{itemize}
\end{frame}


\begin{frame}[fragile]{线性筛}{欧拉筛}
  \begin{itemize}
    \item 如果能让每个合数都只被标记一次就好了
    \pause
    \item 算法实现:
\begin{lstlisting}
inline void getPrime(int n) {
    for (int i = 2; i <= n; ++i) {
        if (!inp[i]) prime[++cnt] = i;
        for (int j = 1; j <= cnt && i * prime[j] <= n; ++j) {
            inp[i * prime[j]] = true;
            if (!(i % prime[j])) break;
        }
    }
}
\end{lstlisting}
    \pause
    \item 时间复杂度:$O(n)$
    \item 非常重要,可以用来线性求解某些\textbf{积性函数}
  \end{itemize}
\end{frame}