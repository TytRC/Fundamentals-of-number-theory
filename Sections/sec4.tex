\section{欧几里得算法}
\subsection{因数}
\begin{frame}[fragile]{因数}
  \begin{theorem}[算术基本定理]
    $\forall A\in \N,A>1 \quad \exists p_1<p_2<p_3<\cdots<p_n,a_i\in \Z^+ \  \mathit{s.t.}\  \prod_{i=1}^n p_i^{a_i}=A$\\
    其中$p_i$是一个质数。这种表示的方法存在,而且是唯一的。
  \end{theorem}
  \pause 
  \begin{itemize}
    \item $1\sim 1e9$的数最多有$1344$个因子。(by 欢神)
    \pause
    \item 求解一个数的所有因数:
    \begin{lstlisting}
vector<int> breakdown(int n) {
  vector<int> res;
  for (int i = 2; i * i <= n; i++) {
    if (n % i == 0) {
      res.push_back(i);
      res.push_back(n / i);
    }
  }
  return res;
}
    \end{lstlisting}
    \item 时间复杂度:$O(\sqrt{n})$
  \end{itemize}
\end{frame}

\begin{frame}[fragile]{因数}
  \begin{itemize}
    \item 求解一个数的所有\textbf{质因数}(即质因数分解):
    \pause
    \begin{lstlisting}
vector<int> breakdown(int n) {
  vector<int> res;
  for (int i = 2; i * i <= n; i++) {
    if (n % i == 0) {
      while (n % i == 0) n /= i;
      res.push_back(i);
    }
  }
  if (n != 1) res.push_back(n);
  return res;
}
    \end{lstlisting}
    \item 时间复杂度:$O(\sqrt{\frac{n}{\ln{n}}})$
    \pause
    \item \textbf{详细介绍}:\href{https://oi-wiki.org/math/pollard-rho/}{分解质因数 - Oi-Wiki} 
  \end{itemize}
\end{frame}

\subsection{最大公约数}
\begin{frame}[fragile]{最大公约数}{GCD}
  最大公约数,是指一组数所有公约数里面最大的一个。\vspace{0.1cm}
  \pause
  \begin{itemize}
    \item $a$和$b$的最大公约数记为:$\gcd(a, b)$
    \pause
    \item \textbf{事实}:
      $$
      \begin{aligned}
        \gcd(a,b)&=\gcd(b,a-b) \\ 
        \gcd(a,b)&=\gcd(a,a\bmod b)
      \end{aligned}
      $$
    \pause
    \item 集合$S=\{\gcd(a_i,\dots,a_n)|1 \leq i\leq n\}$至多有$O(\max(a_i))$个元素。
  \end{itemize}
\end{frame}

\begin{frame}[fragile]{欧几里得算法}
  \begin{itemize}
    \item 算法实现:
    \begin{lstlisting}
int gcd(int a, int b) {
  if (b == 0) return a;
  return gcd(b, a % b);
}
    \end{lstlisting}
    \item 时间复杂度:$O(\log{n})$
    \pause
    \item 求解\textbf{斐波那契数列}相邻两项时达到最坏时间复杂度
    \pause
    \item \lstinline|<algorithm>|库内置了\lstinline|__gcd()|函数
    \pause
    \item 本质上是将将大规模问题转化为小规模问题
    \item 将问题$(a,b)$转化为了$(b,a\bmod b)$
  \end{itemize}
\end{frame}

\begin{frame}[fragile]{多个数的最大公约数}
  求解多个数的最大公约数:
  \begin{itemize}
    \item 可以证明,$\gcd(a,\gcd(b,c))=\gcd(a,b,c)$
    \item 即每次取出两个数求出$\gcd$后再放$\gcd$回去,不会对所需要的答案造成影响。
    \pause
    \begin{lstlisting}
int gcd(int a[], int n) {
  int res = a[1];
  for (int i = 2; i <= n; ++i)
    res = __gcd(res, a[i]);
  return res;
}
    \end{lstlisting}
  \end{itemize}
\end{frame}

\begin{frame}[fragile]{例题}{又是毕业季}
  题目链接:\href{https://www.luogu.com.cn/problem/P1372}{又是毕业季I}
  \begin{block}{题目描述}
    给定$n$和$k$,问在$1\sim n$中选$k$个数使得它们$\gcd$最大,求最大值是多少。\\
    $1 \leq k\leq n\le 1e9$
  \end{block}
  % \vspace{0.3cm}
  \pause
  \begin{exampleblock}{题解}
    \begin{itemize}
      \item \textbf{反向考虑},即考虑一个数$x$,$1\sim n$中有多少数被它整除。
      \pause
      \item 显然为$\lfloor \frac{n}{x} \rfloor$个
      \item 即求$\max\limits_{x}\{[\lfloor \frac{n}{x} \rfloor \geq k]\}$,
      显然答案为$\lfloor \frac{n}{k} \rfloor$
      \pause
      \item 时间复杂度:$O(1)$
    \end{itemize}
  \end{exampleblock}
  \pause
  \begin{lstlisting}
    n, k = map(int, input().split())
    print(int(n / k))
  \end{lstlisting}
  \pause
  同样思想解决的题:\href{https://www.luogu.com.cn/problem/P1414}{又是毕业季II}(双倍经验它不香吗?)。
\end{frame}

\begin{frame}[fragile]{例题}{最大公约数}
  题目链接:\href{https://www.luogu.com.cn/problem/P7243}{最大公约数}
  \begin{block}{题目描述}
    有一个$n \times m$的矩阵$a$。对此矩阵进行变换,定义为将这个矩阵内的所有元素变为其上下左右四个元素(不存在则忽略)及自身的最大公约数。\\询问$a_{x,y}$在进行最少多少次变换之后会变成$1$。如果可以使$a_{x,y}$经过若干次变换变成$1$,输出其中最小的次数;否则输出$-1$。\\
    \vspace{0.3cm}
    \pause
    数据范围:
    $$
    1\le n,m\le 2\times 10^3, 1\le a_{i,j}\le 10^{18}, 1\le x\le n,1\le y\le m
    $$
  \end{block}
  \vspace{0.3cm}
  \pause
  \begin{alertblock}{hint}
    \begin{itemize}
      \item \textbf{事实}:
      $$
      \begin{aligned}
        &\gcd(a,\gcd(b,c))&=\gcd(a,b,c)\\
        &\gcd(\gcd(a,b),\gcd(b,c))&=\gcd(a,b,c)
      \end{aligned}
      $$
      \item 分别考虑“判断可能”和“求解”。
    \end{itemize}
  \end{alertblock}
\end{frame}

\begin{frame}[fragile]{例题}{最大公约数}
  \begin{block}{题解}
    显然,当且仅当所有元素的$gcd$不为$1$时,答案不存在。\\
    \pause
    当答案存在时,我们考虑每一次矩阵变换。
    \begin{itemize}
      \pause
      \item 第一次,$a_{x,y}$由$a_{x,y}$上下左右四个点来更新。\\
      即$a_{x,y}=\gcd(a_{x-1,y},a_{x+1,y},a_{x,y-1},a_{x,y+2})$。
      \item 记$a_{x,y}$周围的四个点为$b_1,b_2,b_3,b_4$。\\
      同时$b_i$也由它们的上下左右四个点来更新
      \pause
      \item 第二次,相当于$a_{x,y}$由每个$b_i$周围的四个点更新。\\
      即相当于$a_{x,y}$由距离它小于$2$的点更新。
      \item 同时$b_i$也是由距离它小于$2$的点更新。
      \pause
      \item 以此类推,假设第$k$次,$a_{x,y}$由距离它小于$k$的点更新。\\
      那么$b_i$也是由距离它小于$k$的点更新。
      \pause
      \item 那么显然第$k+1$次,$a_{x,y}$由距离$b_i$小于$k$的点更新。\\
      即相当于距离它小于$k+1$的点更新。
    \end{itemize}
  \end{block}
\end{frame}

\begin{frame}[fragile]{例题}{最大公约数}
  \begin{block}{题解}
    \begin{itemize}
      \item 即我们归纳证明了,第$k$次,$a_{x,y}$由距离它小于$k$的点更新。
      \item 即,第$k$次变换后,$a_{x,y}$的值变成了所有距它小于等于$k$的点的值的$\gcd$。
      \pause
      \item 题目转化为:从$(x,y)$开始一层层取$\gcd$,直到$\gcd=1$。
      \pause
      \item 从$(x,y)$开始bfs,同时对每个遍历到的元素取$\gcd$,直到$\gcd=1$。
      \item 遍历层数就是答案。
    \end{itemize}
  \end{block}
  
\end{frame}

\subsection{最小公倍数}
\begin{frame}[fragile]{最小公倍数}{LCM}
  \begin{itemize}
    \item \textbf{事实}:
    $$\forall a,b>0,a,b\in \N, a\times b=\gcd(a,b)\times \text{lcm}(a,b)$$
    \item 所以求出俩数的GCD后LCM可在$O(1)$的时间内求出
    \pause
    \item 算法实现:
    \begin{lstlisting}
int lcm(int a, int b) {
  return a * b / __gcd(a, b);
}
    \end{lstlisting}
    \pause
    \item \textbf{多个数的LCM}
    \begin{itemize}
      \item 与多个数的GCD类似
      \pause
      \item 每次取出两个数求LCM后再将LCM放回去,不会对所需要的答案造成影响
    \end{itemize}
  \end{itemize}
\end{frame}

\begin{frame}{例题}{最大公约数和最小公倍数问题}
  题目链接:\href{https://www.luogu.com.cn/problem/P1029}{[NOIP2001 普及组] 最大公约数和最小公倍数问题}
  \begin{block}{题目描述}
    输入两个正整数$x_0$, $y_0$,$(2\leq x_0,y_0\leq 10^5)$,求出满足下列条件的$P$,$Q$的个数:
    \begin{itemize}
      \item $P$,$Q$是正整数。
      \item 要求$P$,$Q$以$x_0$为最大公约数,以$y_0$为最小公倍数。
    \end{itemize}
    试求:满足条件的所有可能的$P$,$Q$的个数。
  \end{block}
  \vspace{0.3cm}
  \pause
  \begin{exampleblock}{题解}
    \begin{itemize}
      \item 根据$P\times Q=\gcd(P,Q)\times \text{lcm}(P,Q)=x_0y_0$,我们可以得到$P\times Q$的值
      \item 枚举$i|(x_0y_0)$,考虑$i,\lfloor \frac{x_0y_0}{i} \rfloor$的最大公约数和最小公倍数是不是分别为$P,Q$即可
      \pause
      \item 时间复杂度:$O(n\log{n})$
    \end{itemize}
  \end{exampleblock}
\end{frame}

\subsection{拓展欧几里得算法}
\begin{frame}[fragile]{拓展欧几里得算法}{简介}
  \begin{itemize}
    \item 拓展欧几里得算法用于求解$ax+by=\gcd(a,b)$的一组可行解
    \pause
    \item 常用于求解\textbf{乘法逆元}
    \pause 
    \item \textbf{裴蜀定理}:该方程组的解一定存在
    \begin{theorem}[裴蜀定理]
      设 $a,b$ 是不全为零的整数,则存在整数 $x,y$, 使得 $ax+by=\gcd(a,b)$。      
    \end{theorem}
    \pause
    \item \textbf{算法思想}:将问题$(a,b)$转化为$(b, a\bmod b)$
    \pause
    \item 要注意负值
    
  \end{itemize}
\end{frame}

\begin{frame}[fragile]{拓展欧几里得算法}{证明}
  \begin{proof}
  \pause
$$
\begin{aligned}
  &ax_1+by_1&=&\gcd(a,b)&\\
  &bx_2+(a\bmod b)y_2&=&\gcd(b,a\bmod b)&
\end{aligned}
$$
\pause 
$$
\begin{aligned}
  &\because    \gcd(a,b)   &=& \gcd(b,a\bmod b) \\
  \pause
  &\therefore  ax_1+by_1   &=& bx_2+(a\bmod b)y_2\\
  \pause
  &\because    a\bmod b    &=& a-(\lfloor\frac{a}{b}\rfloor\times b)\\
  \pause 
  &\therefore  ax_1+by_1   &=& bx_2+(a-(\lfloor\frac{a}{b}\rfloor\cdot b))y_2\\
  &                      &=& ay_2+bx_2-\lfloor\frac{a}{b}\rfloor\cdot by_2\\
  \pause 
  &\therefore  ax_1+by_1   &=& ay_2+b(x_2-\lfloor\frac{a}{b}\rfloor y_2)\\
  \pause
  &\because    a=a,&\;&b=b  \\
  &\therefore  x_1=y_2,&\;&y_1=x_2-\lfloor\frac{a}{b}\rfloor y_2\\
\end{aligned}
$$
  \end{proof}
\end{frame}

\begin{frame}[fragile]{拓展欧几里得算法}{实现}
  \begin{itemize}
    \item $x_1=y_2,y_1=x_2-\lfloor\frac{a}{b}\rfloor y_2$
    \item 其中$x_1,y_1$是问题$(a,b)$的答案\\
    $x_2,y_2$是问题$(b,a\bmod b)$的答案
    \pause 
    \item 算法实现:
    \begin{lstlisting}
    ll exgcd(ll a, ll b, ll& x, ll& y){
      if (!b){
          x = 1, y = 0;
          return a;
      }
      ll gcd = exgcd(b, a % b, y, x);
      y -= a / b * x;    
      return gcd;
    }
    \end{lstlisting}
  \end{itemize}
\end{frame}

\begin{frame}[fragile]{拓展欧几里得算法}{拓展}
  \begin{itemize}
    \item 我们已经知道了不定方程$a∗x+b∗y=c$的一组解为$(x_0,y_0)$那么它的全部解为:
      $$
      \left\{\begin{array}{l}
      x=x_{0}+\frac{b}{\operatorname{gcd}(a, b)} * t \\
      y=y_{0}-\frac{a}{g c d(a, b)} * t
      \end{array}, t \in \mathbb{Z}\right.
      $$
    \pause
    \item 另外,拓欧算法中,以下定理成立:
    \begin{theorem}
      当$a,b>0$且$\gcd(a,b)\neq\min(a,b)$时有$|x|<\frac{b}{\gcd(a,b)}$和$|y|<\frac{a}{\gcd(a,b)}$
    \end{theorem}
    \pause
    \item 此定理告知我们以下两个事实:
    \begin{enumerate}
      \item 扩欧过程中不会发生溢出
      \item 不定方程的某个最小正整数解为$\left(x+\frac{b}{g c d(a, b)}\right) \bmod \frac{b}{g c d(a, b)}$\\
      和$\left(y+\frac{a}{g c d(a, b)}\right) \bmod{\frac{a}{g c d(a, b)}}$。
    \end{enumerate}
  \end{itemize}
\end{frame}

\begin{frame}[fragile]{例题}{同余方程}
  题目链接:\href{https://www.luogu.com.cn/problem/P1082}{[NOIP2012 提高组] 同余方程}
  \begin{block}{题目描述}
    求关于$x$的同余方程 $ax \equiv 1 \pmod {b}$的\textbf{最小正整数解}。\\
    数据保证一定有解。\\
    $2 \leq a,b\leq 2e9$
  \end{block}
  \vspace{0.1cm}
  \pause

  \begin{exampleblock}{题解}
    \begin{itemize}
      \item $ax \equiv 1 \pmod {b}$可以转化为$ax=(-b)y+1$,即$ax+by=1$。
      \pause
      \item 拓展欧几里得裸题。
      \item 注意判断负值。
      \pause
      \item 时间复杂度:$O(\log{a})$
    \end{itemize}
  \end{exampleblock}
  \pause

  \textbf{拓展欧几里得习题推荐}:
  \begin{itemize}
    \item \href{https://www.luogu.com.cn/problem/P1516}{P1516 青蛙的约会 - 洛谷}
    \item \href{https://acm.dingbacode.com/showproblem.php?pid=5512}{5512 Pagodas - HDOJ}(裴蜀定理)
  \end{itemize}
\end{frame}