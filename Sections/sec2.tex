\section{整除与同余}
\subsection{整除与同余}
\begin{frame}{整除基本性质}
\textbf{整除的定义}:若整数$a$除以非零整数$b$,商为整数且余数为零,即$a$能被$b$整除,记做$b|a$,读作:$b$整除$a$或$a$能被$b$整除。$a$叫做$b$的倍数,$b$叫做$a$的约数,或称为因数。
\pause
\begin{itemize}
  \item 若$a|c,b|c$,则$a|(b\pm c)$
  \item 若$a|b$,则对任意的$c(c\neq 0)$,有$a|bc$
  \item 若$a|b$,且$b|c$,则$a|c$
  \item 若$a|bc$,且$(a,c)=1$,则$a|b$
  \item 若$c|a$,且$c|b$,则对于任意整数$m,n$,有$c|(ma+nb)$
  \item 若$a|c,b|c$,且$(a,b)=1$,则$ab|c$
  \item 若$p|bc$,$p$是质数,那么$p|b$或$p|c$至少一个成立
  \item \textbf{带余除法定理}: $\forall a,b>0, a,b\in \N$,$\exists$唯一的数对$q,r$,使$a=bq+r,(0\leq r<b)$。
\end{itemize}
\end{frame}


\begin{frame}{同余基本性质}
\begin{itemize}
  \item 若$a\equiv b\pmod{m},c\equiv d\pmod{m}$,\\则$a\pm b\equiv c\pm d\pmod{m}$
  \item 若$a\equiv b\pmod{m},c\equiv d\pmod{m}$,\\则$a\times b\equiv c\times d\pmod{m}$ \pause
  \item 证明基本思路:$a=k_1m+b,c=k_2m+b$ \pause
  \item 基本用法:求模运算下求解问题(避免大数运算)、快速幂 \pause
  \item \textbf{注意}:除法不满足以上的性质
\end{itemize}
\end{frame}

\subsection{快速幂}
\begin{frame}[fragile]{快速幂}
  求解:$a^b\equiv x\pmod{m},(a,b\leq 1e9)$
  \vspace{0.5cm}
  \pause
\begin{itemize}
  \item 朴素求法:
\begin{lstlisting}
inline ll mpow(ll a, ll b, ll m) {
  ll res = 1;
  for (int i = 1; i <= b; ++i) 
    res = res * a % m;
  return res;
}
\end{lstlisting}
  \item 时间复杂度:$O(b)$
\end{itemize}
\end{frame}


\begin{frame}[fragile]{快速幂}
  求解:$a^b\equiv x\pmod{m},(a,b\leq 1e9)$
  \vspace{0.5cm}
\begin{itemize}
  \item 快速幂
\begin{lstlisting}
  inline ll mpow(ll a, ll b, ll m) {
    ll res = 1;
    while(b) {
      if (b & 1) res = res * a % m;
      a = a * a % m;
      b >>= 1;
    }
    return res;
  }
\end{lstlisting}
  \item 时间复杂度:$O(\log{b})$
\end{itemize}
\end{frame}